
In the new \GLS{ros}-2 world, packages can still be setup with simple python scripts.
Also to setup a robot which can then be visualized in \Gls{gazebo} and/or \Gls{rviz}, the main file format did not change: URFD.
However to simplify the whole process of designing a robot, \textit{xacro}\footnote{xacro is an abbreviation from macro but for xml.} may be used.


\begin{enumerate}
    \item A simple start script can be found under \href{https://github.com/LukyLuke/mse_vt_eeros/blob/main/src/demo_package/launch/simulation.launch.py}{launch/simulation.launch.py}.

    \item The Robot/Motor is defined in the description file \\ \href{https://github.com/LukyLuke/mse_vt_eeros/blob/main/src/demo_package/description/demo_motor_simulation.urdf.xacro}{description/demo\_motor\_simulation.urdf.xacro}.

    \item In the \href{https://github.com/LukyLuke/mse_vt_eeros/tree/main/src/demo_package/src}{/src} and \href{https://github.com/LukyLuke/mse_vt_eeros/tree/main/src/demo_package/include/demo_package}{/include} folder is a simple \GLS{eeros}-Project based on \GLS{ros}-2.
\end{enumerate}


\section[startup]{Startup} \label{sec:demo-startup}

\begin{enumerate}
    \item Install and setup \GLS{ros}-2 as described in the official documentation.
    \item Install and setup \GLS{eeros} as described in the official documentation.
    Until the merge of the Pull-Requests in \ref{app:eeros} and \ref{app:ros-eeros}, use the \textit{ros-2} branches of \textit{eeros} and \textit{ros-eeros}.
    \item Clone the \href{https://github.com/LukyLuke/mse_vt_eeros}{GIT-Repository} into the EEROS-Root-Directory.
    \item Change the value \texttt{EEROS\_INCLUDE\_DIRS} if needed.
    \item Create the \texttt{build-x86/demo\_package} and \texttt{install-x86/demo\_package} folder.
    \item Run cmake inside the build directory with the correct\\
    \texttt{-DCMAKE\_INSTALL\_PREFIX=../../install-x86/demo\_package} parameter.
    \item Build the demo\_package: `make`
    \item Install the demo\_package with \texttt{colcon install --symlink-install}
    \item Start the EEROS-Node\footnote{Probably you have to set \texttt{LD\_LIBRARY\_PATH="../lib/:\$LD\_LIBRARY\_PATH"}}: \texttt{./demo\_package/bin/demo\_package/demo\_package}
    \item Start the package: \texttt{ros2 launch demo\_package simulation.launch.py}
\end{enumerate}

In the console of the \GLS{eeros}-Node all the received and sent out messages are logged.

In \Gls{gazebo} you can press the pause button in the status bar to show that now for this simulation \Gls{gazebo} is the main clock.


\subsection[Rviz2 Config]{Configure RViz2} \label{sec:demo-rviz2-config}

Once started up, in \Gls{rviz} three configurations should be done to see the robot and it's joints:

\begin{enumerate}
    \item Add the TF Transform Hierarchy: Add -> TF -> OK
    \item To show the Joint-Names, check the option \textit{Show Names} in the settings\footnote{The Settings are shown when you expand the section on the left.} of TF.
    \item Add the Robot Model: Add -> RobotModel -> OK
    \item Set the \textit{Description Topic} in the RobotModel to \textit{/robot\_description}.
\end{enumerate}

If everything is running as expected, the \textit{wheel\_joint} should rotate.


\subsection[RQT Config]{Configure RQT} \label{sec:demo-rqt-config}

In RQT you can show the values of the current joint value in contrast to the calculated value:

\begin{enumerate}
    \item Add the topic value \texttt{/joint\_states/position[0]} which is the value \Gls{gazebo} sends out.
    \item Add the topic value \texttt{/set\_joint\_trajectory/points[0]/positions[0]} which is the value from \GLS{eeros}.
\end{enumerate}

If everything is running as expected, the graph should show the two values (actually on top of each other because for some reason the acceleration, velocity and effort values are ignored).


\section[Source]{Source Code} \label{sec:demo-source-code}

In this section the most important parts of the source code and used messages are explained. The Source of this demo package can be found in the \href{https://github.com/LukyLuke/mse_vt_eeros/}{Project Repository}.

\subsection[URDF-Model]{The URDF-Robot-Model} \label{sec:demo-urdf-model}

There are the two sections on the bottom which are for interacting with \Gls{gazebo}.

\subsubsection[Joint-State]{Joint-State Publisher Plugin} \label{sec:demo-urdf-model-joint-state}

The Joint-State Publisher \footnote{See \href{https://github.com/ros-simulation/gazebo_ros_pkgs/blob/ros2/gazebo_plugins/include/gazebo_plugins/gazebo_ros_joint_state_publisher.hpp}{gazebo\_ros\_joint\_state\_publisher.hpp} for the Source-Code and further documentation.}.
For the message type see chapter \ref{sec:demo-subscriber}.

\lstset{language=XML}
\begin{lstlisting}[label=code:xml-urdf-joint-state, caption={[URDF-Section for Joint-States]Loading the \Gls{gazebo}-Plugin and publish \textit{wheel\_joint} to the \textit{/joint\_states} topic.}]
<!-- Publish the position of the given joints to the topic /joint_states -->
<plugin name="gazebo_ros_joint_state_publisher" filename="libgazebo_ros_joint_state_publisher.so">
    <update_rate>10</update_rate>
    <joint_name>wheel_joint</joint_name>
    <ros><namespace>/</namespace></ros>
</plugin>
\end{lstlisting}

\begin{description}
    \item[joint\_name] List of Joint-Names to publish, one tage for each joint.
    \item[update\_rate] \textit{(default: 100.0)} Number of milliseconds to publish to the Joint-States topic (Hz).
    \item[namespace] \textit{(default: /)} The Root-Topic under which the /joint-state topic will be published.
\end{description}


\subsubsection[Joint-Pose]{Joint-Pose Subscriber Plugin} \label{sec:demo-urdf-model-joint-pose}

The Joint-Trajectory-Pose Subscriber \footnote{See \href{https://github.com/ros-simulation/gazebo_ros_pkgs/blob/ros2/gazebo_plugins/include/gazebo_plugins/gazebo_ros_joint_pose_trajectory.hpp}{gazebo\_ros\_joint\_pose\_trajectory.hpp} for the Source-Code and further documentation.} which reads messages from a topic and applies the values to the given joints.
For the message type see chapter \ref{sec:demo-publisher}.

\lstset{language=XML}
\begin{lstlisting}[label=code:xml-urdf-joint-pose, caption={[URDF-Section for Joint-Trajectory Poses]Loading the \Gls{gazebo}-Plugin and subscribe to the \textit{set\_joint\_trajectory} topic to receive new positions for the given joints in the message.}]
<!-- Reads a trajectory_msgs/msg/JointTrajectory from /set_joint_trajectory and uses that for a new joint state -->
<plugin name="gazebo_ros_joint_pose_trajectory" filename="libgazebo_ros_joint_pose_trajectory.so">
    <update_rate>10</update_rate>
    <ros><namespace>/</namespace></ros>
</plugin>
\end{lstlisting}

\begin{description}
    \item[update\_rate] \textit{(default: 100.0)} Number of milliseconds to read messages from the topic (Hz).
    \item[namespace] \textit{(default: /)} The Root-Topic under which the /set\_joint\_trajectory topic will be read.
\end{description}



\subsection[Subscriber]{Joint-State Subscriber} \label{sec:demo-subscriber}

For demonstration purposes, a simple \texttt{JointStateSubscriber}\footnote{\url{https://github.com/LukyLuke/mse_vt_eeros/blob/main/src/demo_package/include/demo_package/JointStateSubscriber.hpp}} is implemented in an \GLS{eeros} Project.
It simply receives a \texttt{sensor\_msgs/msg/JointState} ROS-Message and transforms it into the internal \texttt{JointState} type.

\subsubsection[JointState-Message]{The ROS-JointState Message Format} \label{sec:deom-subscriber-message}

A \texttt{JointState}-Message\footnote{See \href{https://docs.ros2.org/latest/api/sensor_msgs/msg/JointState.html}{JointState.msg} for more documentation.} consists of four fields, which are all lists where the indexes corresponds:

\begin{description}
    \item[name] \textit{(string[])} List of all Joints which are published
    \item[position] \textit{(i64[])} The position of the joint in $rad$ or $meter$
    \item[velocity] \textit{(i64[])} The current velocity in $rad/s$ or $m/s$
    \item[effort] \textit{(i64[])} The applied force in $Nm$ or $N$
\end{description}


\subsection[Publisher]{Trajectory-Pose Subscriber} \label{sec:demo-publisher}

To publish a changed value, a \texttt{JointTrajectoryPublisher}\footnote{\url{https://github.com/LukyLuke/mse_vt_eeros/blob/main/src/demo_package/include/demo_package/JointTrajectoryPublisher.hpp}} is implemented for demonstration purposes.
It simply converts the double from a signal into a\\
\texttt{trajecotry\_msgs/msg/JointTrajectory} ROS-Message and sends it out to the topic\footnote{The acceleration, velocity and effort values are not used in \Gls{gazebo}, or I understand something totally wrong.}.
Thereby the \textit{acceleration}, \textit{velocity} and \textit{effort} is set to a static value.
This because the internally used \texttt{JointState} from the \texttt{JointStateSubscriber} could not be used easily as a Constant or any other EEROS-Type.

\subsubsection[JointTrajectoryPose-Message]{The ROS-JointTrajectory Message Format} \label{sec:deom-publisher-message}

A \texttt{JointTrajectory}-Message\footnote{See \href{https://docs.ros2.org/latest/api/trajectory_msgs/msg/JointTrajectory.html}{JointTrajectory.msg} for more documentation.} consists of two fields, which are all lists where the indexes corresponds:

\begin{description}
    \item[joint\_names] \textit{(string[])} List of all Joints which are published
    \item[points] \textit{(JointTrajectoryPoint[])} List of Points which should be reached
\end{description}

The \texttt{JointTrajectoryPoint} ROS-Message consists of four lists which have to correspond to the \textit{joint\_names} in the \texttt{JointTrajectory} message:

\begin{description}
    \item[positions] \textit{(i64[])} The position of the joint in $rad$ or $meter$
    \item[velocities] \textit{(i64[])} The current velocity in $rad/s$ or $m/s$
    \item[accelerations] \textit{(i64[])} The applied rate of change in velocity in $rad/sec^2$ or $m/sec^2$
    \item[effort] \textit{(i64[])} The applied force in $Nm$ or $N$
\end{description}



