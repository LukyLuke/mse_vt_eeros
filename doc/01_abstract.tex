
\section*{Abstract} \label{sec:abstract}

The main defined goal of the Project was to have a ``Motor'' in \GLS{eeros} which is simulated/visualized in \Gls{gazebo}.

Just when starting to setup the whole toolchain, \GLS{eeros}, \GLS{ros} and \GLS{gazebo}, it showed that \GLS{eeros} is not working together with the new \GLS{ros}-2.
Current Linux distributions do not support \GLS{ros}-1 anymore out of the box because it will be End of Life in 2025.
This was the point where the first goal changed to port \GLS{eeros} to \GLS{ros}-2.
The second goal then was to have a motor simulation as defined by the initial task definition.

Due to the fact that I did neither know \GLS{eeros} nor \GLS{ros}, to change all the code to work properly with \GLS{ros}-2 took some time.
Also the documentation of both projects, in my point of view, lack of description on how they should be used.
For \GLS{ros}-1 there exists quite a lot of tutorials, Q\&A's and Code examples; \GLS{ros}-2 seems not yet to be at that point.

One of the biggest challenges was therefore to understand the way \GLS{ros}-1 was working and how this changed in \GLS{ros}-2.
The main change in \GLS{ros}-2 is, that the whole application spins and not only a single node.
The effect of that behavior: As soon as a subscriber is created, the whole \GLS{eeros} program is locked.
However, for such applications there was a Executor introduced in \GLS{ros}-2, which does similar things than the \GLS{eeros}-Executor.

To not refactor \GLS{eeros} completely, a \GLS{ros}-2 Executor is used as soon as a subscriber is is used.
All received messages are stored in a queue and processed as soon as the \GLS{eeros}-Executor calls for them.
If a \GLS{ros}-Topic is the main clock, the process is inverse: The \GLS{ros}-2 Executor triggers the \GLS{eeros}-Executor for processing the blocks.

For having a motor simulated in \Gls{gazebo}, \Gls{gazebo} has to read values from a topic and apply them to the physics simulation.
For such things \Gls{gazebo} has to be extended with plugins.
Although there is a third-party plugin available for a motor simulation, that one is not yet ported to \GLS{ros}-2.
Therefore a different plugin is used which takes values for a joint and simulates the movement.
This movement is then written back to an other topic which is used in \GLS{eeros} again for calculation.
This is not a real ``Motor-Simulation'' as it was thought to be \textit{(acceleration, velocity and efforts are not used in the physics)}, but it shows the concept of how \GLS{eeros} can work together with \Gls{gazebo}.
