
\section*{Abstract} \label{sec:abstract}

Just when starting to setup the whole toolchain \GLS{eeros}, \GLS{ros} and \GLS{gazebo} it showed that \GLS{eeros} only works with an old version of \GLS{ros}-1.
Current Linux distributions do not support \GLS{ros}-1 anymore out of the box.
This was the point where the first goal was to port \GLS{eeros} to \GLS{ros}-2 and second goal to have a working package.

This process took quite long, due to the fact that I did neither know \GLS{eeros} nor \GLS{ros}.
Also the documentation of both projects, in my point of view, lack of description on how they should be used.
This was quite good with lot of code and tutorials for \GLS{ros}-1, but \GLS{ros}-2 seems not yet to be at that point.

One of the biggest challenges was therefore to understand the way \GLS{ros}-1 was working and how this changed in \GLS{ros}-2.
The main change there is, that in \GLS{ros}-2 the whole application spins and not only a node anymore.
The effect was that, as soon as a subscriber was created, the whole \GLS{eeros} program was locked.
However, there was a new Executor introduced in \GLS{ros}-2, which does similar things than the \GLS{eeros}-Executor.

As a quick fix to solve the problem and not refactor \GLS{eeros} completely, a \GLS{ros}-2 Executor is used for all Subscriptions.
All received Messages are stored in a queue and processed as soon as the \GLS{eeros}-Executor calls for them.
If a \GLS{ros}-Topic is the main clock, the process is inverse: The \GLS{ros}-2 Executor triggers the \GLS{eeros}-Executor for processing.

The main goal of the Project, to have a ``Motor'' in \GLS{eeros} which is simulated in \Gls{gazebo} was not reached.
For this to work as expected, there should be a plugin in \Gls{gazebo} which would take ``power'' or something like that from a topic and then simulate rotation.
Although there is a third-party plugin available, that one was also not ported yet to \GLS{ros}-2.
Therefore only a package for demonstration purposes was created.
