
\section[Definition of task]{Definition of task} \label{sec:definition-of-task}

\textit{[German]} Eine bestehende Regelung f\"ur einen Motor soll wahlweise auf realer Hardware und in einer passenden Simulationsumgebung laufen gelassen werden k\"onnen.
Ein spezielles Augenmerk soll auf einer nahtlosen Integration in das bestehende \GLS{eeros} Framework gelegt werden und mit passenden Werkzeugen soll alles m\"oglichst einfach und automatisiert gestartet und demonstriert werden k\"onnen.
Eine komplexere Applikation mit einem kleinen Delta-Roboter soll als zweites Demonstrationsobjekt dienen.

\textit{[English]} It should be possible to run an already existing control system for a simple motor even on real hardware or in a simulation.
The main focus lies on the integration in \GLS{eeros} and some simple tools around it to automate everything for demonstration purposes.
A more complex application based on a delta robot should be a second objective for demonstration.


\section[Setup]{Project Setup} \label{sec:project-setup}

There are mainly three components involved in the whole setup:

\begin{description}
    \item[\GLS{eeros}] is the main part where the controlling should happen.
    \item[\GLS{ros}-2] is the middleware where all messages are handled through topics.
    \item[\Gls{gazebo}] is a physics simulation where the motor/robot is simulated and visualized.
\end{description}

For controlling and demonstration purposes the following tools can be used:

\begin{description}
    \item[Rviz2] is be used to just show the robot and all it's joints.
    \item[RQT] can be used to show values from a topic in a graph, explore the topics and more.
    \item[joint\_state\_publisher\_gui] can be used to control the joints.
\end{description}

How to setup, the source code and a simple package is published under \href{https://github.com/LukyLuke/mse_vt_eeros}{Github: LukyLuke/mse\_vt\_eeros}.

